\documentclass[a4paper,11pt]{article}
\usepackage[english,russian]{babel}
\usepackage[utf8]{inputenc}
\usepackage{a4wide}
\usepackage{amsmath}
\usepackage{float}
\usepackage{graphicx}
\usepackage{indentfirst}
\usepackage{titlesec}
\makeatletter
\renewcommand{\@seccntformat}[1]{}
\makeatother
\date{}
\begin{document}

\title{Смрад}
\maketitle
\tableofcontents

\part{Магия}
\section{Суть магии в целом}
Частички древнего могущества текут в жилах людей. С каждым веком магия слабеет, и лишь немногие семьи в силах ее сохранить. Иногда
способности к магии проявляются и у простых крестьян, будь то из-за прихотей природы наследования или из-за других столкновений с магией былого.

У этих линий силы, текущих через поколения людей, бывают особенности. Они подталкивают людей к тому, чтобы высвобождать энергию
какими-то определенными способами, или каким-то неожиданным образом. Из-за подобных особенностей может даже меняться внешний вид носителя силы.
От их наличия, однако, не зависит способность человека обучаться и пользоваться традиционными школами магии.

Носители силы обычно осознают себя таковыми еще с юных лет, ведь сила проявляет себя. Если не научиться ее использовать, она станет 
отравлять тело подобно яду, приводя к рвоте, появлению волдырей и прочим проблемам, все из которых не описать.

Уровень силы в человеке, впрочем, определяет лишь потолок, до которого он может дорасти. Простейшая магия доступна всем, в ком 
есть хоть малая и незаметная частичка древнего могущества, а тем, кто чувствует в себе хоть какую-то силу, потребуются годы,
чтобы достичь пика своих способностей. Сильнейшие же маги лишь к старости могут раскрыть свой потенциал целиком.
\section{Известные линии силы}
\begin{itemize}
	\item Семья Де Ремар, ткачи, сияющая кровь
	\item Семья Де Серра, эскулапы, гниющие
	\item Заболевшие чумой, обманутая сила
\end{itemize}


\part{География}
\section{Авалон}
\begin{itemize}
	\item Малмут, столица, ходячий город
	\item Высокие столпы, библиотека
\end{itemize}

\part{Главные герои}
\section{Холлен Де Ремар, пустой принц}
\textit{Когда-то наша семья правила лишь графством. Пусть это снова будет так. Я готов жертвовать больными городами, но я никогда пожертвую своей честью.}
\section{Кристофер Фрейн, рука порядка}
\textit{Мы всегда знаем, какой поступок правилен. Нас никто не поблагодарит, но мы должны взвалить на плечи любую ношу - будь то тяжелый труд или грех перед господом.}
\section{Этера Де Серра, надежда чумных}
\textit{Новый путь лучше старого, но люди боятся его. Лучшее решение поначалу всегда невозможным или бесчеловечным. Я должна смотреть сквозь эту пелену предубеждений.}


\end{document}
