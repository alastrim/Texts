\documentclass[10pt,twoside,twocolumn,openany]{book}
\usepackage[bg-letter]{../lib/dnd}
\usepackage[T2A]{fontenc}
\usepackage[utf8]{inputenc}
\usepackage[english,ngerman,main=russian]{babel}
\usepackage{gentium}
\begin{document}
\fontfamily{gentium}\selectfont
\tableofcontents

\clearpage
\part{Создание Персонажа}
\onecolumn
\section{Лист Персонажа}
\begin{monsterbox}{Кристофер Фрейн}
	\monsterquote{Мы всегда знаем, какой поступок правилен. Нас никто не поблагодарит, но мы должны взвалить 
		на плечи любую ношу - будь то тяжелый труд или грех перед господом.}
	\begin{monstersection}{Характеристики}
		\stats[STR = \stat{2}, DEX = \stat{4}, CON = \stat{2}, INT = \stat{1}, WIT = \stat{2}, WIL = \stat{10}]
	\end{monstersection}
	\begin{monstersection}{Защита}
		\basics[speed = 5, dodge = 8, armor = 12, block = 2]
	\end{monstersection}
	\begin{monstersection}{Атаки}
	\end{monstersection}
	\begin{monstersection}{Заклинания}
	\end{monstersection}
	\begin{monstersection}{Навыки}
		\monsteraction{Железная стена}{Один раз за бой персонаж может гарантированно уклониться от физической атаки. Это решение 
			можно принять после броск атаки врага, но до броска урона.}
		\monsteraction{Несгибаемая воля (1)}{Персонаж может использовать волю для поглощения физического урона. Воля поглощается по единичной цене.}
	\end{monstersection}
	\begin{monstersection}{Инвентарь}
		\weapon[name = Меч, hit = + 1, dmg = - 1, special = парирующее]
		\weapon[name = Глефа, hit = + 3, dmg = + 2, special = {длинное, двуручное}]
		\weapon[name = Люцерн, hit = - 1, dmg = + 4, special = {пробивающее, двуручное}]
		\weapon[name = Кинжал, hit =  -4, dmg = + 1, special = пробивающее]
		\armor[name = Кольчуга, dodge = + 4, armor = + 8, block = 2] 
	\end{monstersection}

\end{monsterbox}
\twocolumn
\clearpage

\section{Создание персонажа}
\lipsum[1]
\subsection{Первый уровень}
\lipsum[1]
\subsection{Повышение уровня}
\lipsum[1]

\part{Бой}
\section{Общий ход боя}
\lipsum[1]
\section{Атака}
\lipsum[1]
\section{Урон}
\lipsum[1]
\section{Здоровье}
\lipsum[1]
\section{Типы действий в бою}
\lipsum[1]

\part{Навыки}
\section{Общая информация}
\lipsum[1]
\section{Общие}
\lipsum[1]
\section{Воинские}
\lipsum[1]
\section{Магические}
\lipsum[1]

\part{Экипировка}
\section{Общая информация}
\lipsum[1]
\section{Оружие}
\lipsum[1]
\section{Доспехи}
\lipsum[1]

\part{Заклинания}
\section{Общая информация}
\lipsum[1]
\section{\textrm{I} Круг}
\lipsum[1]
\section{\textrm{II} Круг}
\lipsum[1]
\section{\textrm{III} Круг}
\lipsum[1]

\end{document}
