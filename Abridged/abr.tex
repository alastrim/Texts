\documentclass[10pt,twoside,twocolumn,openany]{book}
\usepackage[bg-letter]{../lib/dnd}
\usepackage[T2A]{fontenc}
\usepackage[utf8]{inputenc}
\usepackage[english,ngerman,main=russian]{babel}
\usepackage{gentium}
\begin{document}
\fontfamily{gentium}\selectfont
\tableofcontents

\clearpage
\part{Создание Персонажа}
\onecolumn
\section{Лист Персонажа}
\begin{monsterbox}{Кристофер Фрейн}
	\textit{Мы всегда знаем, какой поступок правилен. Нас никто не поблагодарит, но мы должны взвалить 
	на плечи любую ношу - будь то тяжелый труд или грех перед господом.}\\
	\hline
	\stats[STR = \stat{8}, DEX = \stat{9}, CON = \stat{10}, INT = \stat{35}, WIT = \stat{38}, WIL = \stat{50}]
	\hline
	\basics[armorclass = 12, hitpoints = 16 (3d8+3), speed = 50 ft]
	\hline
	\details[languages = {Лисп, Эрланг},]
	\hline
	\begin{monsteraction}[Суперсилы]
		У этого монстра есть суперсилы!
	\end{monsteraction}
	\monstersection{Навыки}
	\begin{monsteraction}[Стена стали]
		Вы можете отбить первое попадание по себе в бою.
	\end{monsteraction}\\
	\begin{monsteraction}[Несгибаемая воля (1)]
		Вместо выносливости можно тратить волю.
	\end{monsteraction}
\end{monsterbox}
\twocolumn
\clearpage

\section{Создание персонажа}
\lipsum[1]
\subsection{Первый уровень}
\lipsum[1]
\subsection{Повышение уровня}
\lipsum[1]

\part{Бой}
\section{Общий ход боя}
\lipsum[1]
\section{Атака}
\lipsum[1]
\section{Урон}
\lipsum[1]
\section{Здоровье}
\lipsum[1]
\section{Типы действий в бою}
\lipsum[1]

\part{Навыки}
\section{Общая информация}
\lipsum[1]
\section{Общие}
\lipsum[1]
\section{Воинские}
\lipsum[1]
\section{Магические}
\lipsum[1]

\part{Экипировка}
\section{Общая информация}
\lipsum[1]
\section{Оружие}
\lipsum[1]
\section{Доспехи}
\lipsum[1]

\part{Заклинания}
\section{Общая информация}
\lipsum[1]
\section{\textrm{I} Круг}
\lipsum[1]
\section{\textrm{II} Круг}
\lipsum[1]
\section{\textrm{III} Круг}
\lipsum[1]

\end{document}
